% This is LLNCS.DOC the documentation file of
% the LaTeX2e class from Springer-Verlag
% for Lecture Notes in Computer Science, version 2.4
\documentclass{llncs}
\usepackage{llncsdoc}
\usepackage[spanish]{babel}
%
\begin{document}
\markboth{Sistemas de Recuperaci\'on de Informaci\'on}{Aplicaciones del agrupamiento y de la clasificación en la RI en la Web}
\thispagestyle{empty}
\begin{flushleft}
\LARGE\bfseries Sistemas de Recuperaci\'on de Informaci\'on\\[2cm]
\end{flushleft}
\rule{\textwidth}{1pt}
\vspace{2pt}
\begin{flushright}
\Huge
\begin{tabular}{@{}l}
Aplicaciones del\\
agrupamiento y\\ 
de la clasificación\\ 
en la recuperaci\'on\\ 
de informaci\'on\\
en la Web\\[6pt]
\end{tabular}
\end{flushright}
\rule{\textwidth}{1pt}
\vfill
\begin{flushleft}
\large\itshape
\begin{tabular}{@{}l}
{\large\upshape\bfseries Autores}\\[8pt]
Laura Victoria Riera P\'erez\\[5pt]
Marcos Manuel Tirador del Riego
\end{tabular}
\end{flushleft}

%
\newpage
\pagenumbering{gobble}
\tableofcontents
\thispagestyle{empty}

\newpage
\pagenumbering{arabic}

\section{Agrupamiento}

\begin{itemize}
\item Aprendizaje no supervisado

\item Problema que resuelve
\end{itemize}

\subsection{Algunas definiciones}

\begin{itemize}
\item Flat clustering

\item Hierarchical clustering

\item Hard clustering

\item Soft clustering

\item Hip\'otesis de agrupamiento

\item Cardinalidad
\end{itemize}

\subsection{Flat clustering}

\begin{itemize}
\item Medida de similitud:

\item Medidas de evaluacion:
\begin{itemize}
	\item Criterio interno de calidad
	
	\item Criterio interno de calidad
	
	\item Pureza
	
	\item \'Indice de frontera?
	
	\item Medida F
\end{itemize}

\item Algoritmos:
\begin{itemize}
	\item K-means
	\item EM (generalizaci\'on de K-means)
\end{itemize}
\end{itemize}


\subsection{Hierarchical clustering}

\begin{itemize}
\item Hierarchical agglomerative clustering

\item Medidas de similitud:
\begin{itemize}
	\item Single link clustering
	\item Complete link clustering
	\item Centroid clustering
\end{itemize}

\item Evaluaci\'on de calidad:
\begin{itemize}
\item Group average link
\item M\'etodo de Ward
\end{itemize}

\item Divisive clustering

\item Cluster labeling

\item Algoritmos:
\begin{itemize}
	\item Algoritmo HAC
	\item Divisive Clustering
\end{itemize}
\end{itemize}

\subsection{Aplicaciones a la RI} 
\begin{itemize}
	\item Search result clustering
	
	\item Scatter-Gather
	
	\item Collection clustering
	
	\item Language modeling
	
	\item Cluster-based retrieval
\end{itemize}

\subsection{Ventajas} 

\subsection{Desventajas} 

\section{Clasificaci\'on}

\begin{itemize}
	\item Aprendizaje supervisado
	
	\item Problema que resuelve
\end{itemize}

\begin{itemize}
\item Rule-based classification

\item Statistical classification

\item Feature selection

\item Medidas de evaluaci\'on:
\begin{itemize}
	\item Fitting
	\item Precisi\'on
	\item Recobrado
	\item Medida F (balanceada)
	\item Classification accuracy
\end{itemize}

\item Algoritmos:
\begin{itemize}
	\item Naive Bayes
	\item K-Nearest Neighbours
\end{itemize}
\end{itemize}

\subsection{Aplicaciones a la RI}
\begin{itemize}
	\item Standing queries
	\item Spam filtering
\end{itemize}

\subsection{Ventajas} 

\subsection{Desventajas}

\section{Agrupamiento vs. Clasificaci\'on}

\section{Ejemplos de aplicación}



\end{document}
