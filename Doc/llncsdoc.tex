% This is LLNCS.DOC the documentation file of
% the LaTeX2e class from Springer-Verlag
% for Lecture Notes in Computer Science, version 2.4
\documentclass{llncs}
\usepackage{llncsdoc}
\usepackage[spanish]{babel}
%
\begin{document}
\markboth{Sistemas de Recuperaci\'on de Informaci\'on}{Aplicaciones del agrupamiento y de la clasificación en la RI en la Web}
\thispagestyle{empty}
\begin{flushleft}
\LARGE\bfseries Sistemas de Recuperaci\'on de Informaci\'on\\[2cm]
\end{flushleft}
\rule{\textwidth}{1pt}
\vspace{2pt}
\begin{flushright}
\Huge
\begin{tabular}{@{}l}
Aplicaciones del\\
agrupamiento y\\ 
de la clasificación\\ 
en la recuperaci\'on\\ 
de informaci\'on\\
en la Web\\[6pt]
\end{tabular}
\end{flushright}
\rule{\textwidth}{1pt}
\vfill
\begin{flushleft}
\large\itshape
\begin{tabular}{@{}l}
{\large\upshape\bfseries Autores}\\[8pt]
Laura Victoria Riera P\'erez\\[5pt]
Marcos Manuel Tirador del Riego
\end{tabular}
\end{flushleft}

%
\newpage
\pagenumbering{gobble}
\tableofcontents
\thispagestyle{empty}

\newpage
\pagenumbering{arabic}

\section{Características del agrupamiento y la clasificación}

\section{Principales métodos aplicados a la Recuperación de Información}

\section{Ejemplos de aplicación}

\section{Ventajas y desventajas}- 

\end{document}
