\documentclass[t,compress,10pt,xcolor=dvipsnames]{beamer}

\usepackage{lmodern}
\usepackage[labelformat=empty,labelsep=none]{subfig}
\usepackage{caption}
\usepackage{float}
\usepackage{wrapfig}

\newcommand*\oldmacro{}%
\let\oldmacro\insertshorttitle% 
\renewcommand*\insertshorttitle{%
\oldmacro\hfill%
\insertframenumber\,/\,\inserttotalframenumber}

\definecolor{UniDunkel}{RGB}{73,142,137}
\definecolor{UniHell}{RGB}{142,184,182}

\setbeamertemplate{blocks}[rounded][shadow=true]
\setbeamercolor{structure}{fg=UniDunkel}

\setbeamercolor{block body}{parent=normal text,use=block title,bg=block title.bg!10!bg}
\setbeamercolor{block body alerted}{parent=normal text,use=block title alerted,bg=block title alerted.bg!10!bg}
\setbeamercolor{block body example}{parent=normal text,use=block title example,bg=block title example.bg!10!bg}

\setbeamercolor*{palette primary}{use=structure,fg=white,bg=UniDunkel!115}
\setbeamercolor*{palette secondary}{use=structure, fg=UniHell,  bg=UniHell}
\setbeamercolor*{palette tertiary}{use=structure,fg=white,bg=UniDunkel!115}
\setbeamercolor*{palette quaternary}{fg=Orange}

\setbeamercolor*{sidebar}{use=structure,bg=structure.fg}
\setbeamercolor*{palette sidebar primary}{use=structure,fg=structure.fg!10}
\setbeamercolor*{palette sidebar secondary}{fg=white}
\setbeamercolor*{palette sidebar quaternary}{fg=white}
\setbeamercolor*{titlelike}{parent=palette primary}

\setbeamercolor*{fine separation line}{}

\setbeamercolor{block title}{use=structure,fg=white,bg=structure.fg!0!red}
\setbeamercolor{block title alerted}{use=alerted text,fg=white,bg=alerted text.fg!0!white}
\setbeamercolor{block title example}{use=example text,fg=white,bg=example text.fg!0!white}

\useoutertheme{miniframes}
\useinnertheme{circles}
\setbeamertemplate{footline}[frame number]

\beamertemplatenavigationsymbolsempty

\usepackage{textpos} 
\usepackage{mathptmx}
\usepackage{anyfontsize}
\usepackage{t1enc}
\usepackage{multicol}
\usepackage{lmodern}

%\addtobeamertemplate{frametitle}{}{%
%\begin{textblock*}{100mm}(11cm,-1cm)
%	\includegraphics[height=1cm]{logo.png}
%\end{textblock*}}

%\titlegraphic{
%	\includegraphics[width=5cm]{backg}
%}

\title{\textbf{Agrupamiento y clasificaci\'on en la recuperaci\'on de informaci\'on en la web}}

%\author{Integrantes:\\
%	Marcos Manuel Tirador del Riego\\ 
%	Laura Victoria Riera P\'erez\\
%	Leandro Rodr\'iguez Llosa}
%\institute{Ciencias de la computaci\'on}
\date{}

%\usepackage{tikz}
%\titlegraphic { 
%	\begin{tikzpicture}[remember picture]
%		\node[left=0.2cm] at (current page.30){
%			\includegraphics[width=10cm]{backg}
%		};
%	\end{tikzpicture}
%}

\setbeamercolor{block body}{bg=Emerald,fg=Emerald}



\begin{document}
	
	\begin{frame}
		\begin{center}
			\begin{block}{}
				\centering
				\Large\textcolor{white}{\textbf{Agrupamiento y clasificaci\'on en la recuperaci\'on de informaci\'on en la web}}
			\end{block}
		
		\vspace{0.5em}
		\includegraphics[width=3cm]{clustering.jpg}
		
		\vspace{0.5em}
		\footnotesize
		Integrantes:\\
			Marcos Manuel Tirador del Riego\\ 
			Laura Victoria Riera P\'erez
		
		\vspace{0.7em}
		\tiny	
		Tercer año. Ciencias de la computaci\'on. Universidad de La Habana. Cuba.
		
		\vspace{0.7em}
		\scriptsize
		Noviembre, 2022
		\end{center}
	\end{frame}

	\begin{frame}[allowframebreaks]{\'Indice general}
		\tableofcontents[sections={1}]
		\framebreak
		\tableofcontents[sections={2-4}]
	\end{frame}


	\section{Agrupamiento}
	\frame
	{
		\frametitle{Agrupamiento}
	}
	
	\subsection{Medidas de similitud}
	\frame{
		\frametitle{Medidas de similitud}
	}
	
	\subsection{Medidas de evaluaci\'on}
	\frame
	{
		\frametitle{Medidas de evaluaci\'on}	
	}

	\subsection{Agrupamiento particionado}
	\frame
	{
		\frametitle{Agrupamiento particionado}
	}
	
	\frame
	{
		\frametitle{K-means}
	}

	\subsection{Agrupamiento jer\'arquico}
	\frame
	{
		\frametitle{Agrupamiento jer\'arquico}
	}
	
	
	\subsubsection{Agrupamiento jer\'arquico aglomerativo}
	\frame
	{
		\frametitle{Agrupamiento jer\'arquico aglomerativo}
	}

	\frame
	{
		\frametitle{Medidas de similitud para cl\'usteres en HAC}
	}	
	
	\frame
	{
		\frametitle{Algoritmo HAC}
	}	

	\subsubsection{Agrupamiento jer\'arquico divisivo}
	\frame
	{
		\frametitle{Agrupamiento jer\'arquico divisivo}
	}

	\subsection{Ventajas}
	\frame
	{
		\frametitle{Ventajas}
	}

	\subsection{Desventajas}
	\frame
	{
		\frametitle{Desventajas}
	}

	\subsection{Ejemplos de aplicaci\'on}
	\frame
	{
		\frametitle{Ejemplos de aplicaci\'on}
	}

	\section{Clasificaci\'on}
	\frame
	{
		\frametitle{Clasificaci\'on}
	}
	
	\subsection{Naive Bayes}
	\frame{
		\frametitle{Naive Bayes}
	}

	\subsection{Feature Selection}
	\frame{
		\frametitle{Feature Selection}
	}

	\subsection{K Nearest Neighbor}
	\frame{
		\frametitle{K Nearest Neighbor}
	}
	
	\subsection{Medidas de evaluaci\'on}
	\frame
	{
		\frametitle{Medidas de evaluaci\'on}	
	}
	
	\subsection{Ventajas}
	\frame
	{
		\frametitle{Ventajas}
	}
	
	\subsection{Desventajas}
	\frame
	{
		\frametitle{Desventajas}
	}
	
	\subsection{Aplicaciones en la Recuperaci\'on de la Informaci\'on}
	\frame
	{
		\frametitle{Aplicaciones en la Recuperaci\'on de la Informaci\'on}
	}
	
	\subsection{Otros ejemplos de aplicaci\'on}
	\frame
	{
		\frametitle{Otros ejemplos de aplicaci\'on}
	}

	\section{Conclusiones}
	\frame
	{
		\frametitle{Conclusiones}
	}

	\section{Referencias}
	\frame
	{
		\frametitle{Referencias}
	}
\end{document} 